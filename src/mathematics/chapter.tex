\section{Mathematics}

\subsection{Trigonometry}
\subsubsection{Sum - difference identities}
\vspace{-0.6cm}
\begin{align*}
  \sin(u \pm v) &= \sin(u)\cos(v) \pm \cos(u)\sin(v) \\
  \cos(u \pm v) &= \cos(u)\cos(v) \mp \sin(u)\sin(v) \\
  \tan(u \pm v) &= \dfrac{\tan(u) \pm \tan(v)}{1 \mp \tan(u) \tan(v)}
\end{align*}

\subsubsection{Sum to product identities}
\vspace{-0.6cm}
\begin{align*}
  \cos(u) + \cos(v) &= 2\cos(\dfrac{u + v}{2}) \cos(\dfrac{u - v}{2}) \\
  \cos(u) - \cos(v) &= -2\sin(\dfrac{u + v}{2}) \sin(\dfrac{u - v}{2}) \\
  \sin(u) + \sin(v) &= 2\sin(\dfrac{u + v}{2}) \cos(\dfrac{u - v}{2}) \\
  \sin(u) - \sin(v) &= 2\cos(\dfrac{u + v}{2}) \sin(\dfrac{u - v}{2})
\end{align*}

\subsubsection{Product identities}
\vspace{-0.6cm}
\begin{align*}
  \cos(u)\cos(v) &= \dfrac{1}{2}[\cos(u + v) + \cos(u - v)] \\
  \sin(u)\sin(v) &= -\dfrac{1}{2}[\cos(u + v) - \cos(u - v)] \\
  \sin(u)\cos(v) &= \dfrac{1}{2}[\sin(u + v) + \sin(u - v)]
\end{align*}

\subsubsection{Double - triple angle identities}
\vspace{-0.6cm}
\begin{align*}
  \sin(2u) &= 2\sin(u)\cos(u) \\
  \cos(2u) &= 2\cos^2(u) - 1 = 1 - 2\sin^2(u) \\
  \tan(2u) &= \dfrac{2\tan(u)}{1-\tan^2(u)} \\
  \sin(3u) &= 3\sin(u) - 4\sin^3(u) \\
  \cos(3u) &= 4\cos^3(u) - 3\cos(u) \\
  \tan(3u) &= \dfrac{3\tan(u) - \tan^3(u)}{1 - 3\tan^2(u)}
\end{align*}

\subsection{Sums}
\vspace{-0.6cm}
\begin{align*}
  &\sum\limits_{i = a}^{b} c^i = \dfrac{c^{b + 1} - c^{a}}{c - 1},\ c \neq 1
  &&\sum\limits_{i = 0}^{n} i c^i = \dfrac{nc^{n + 2} - (n + 1)c^{n + 1} + c}{(c - 1)^2},\ c \neq 1 \\
  &\sum\limits_{i = 1}^{n} i = \dfrac{n(n + 1)}{2}
  &&\sum\limits_{i = 1}^{n} i^2 = \dfrac{n(n + 1)(2n + 1)}{6} \\
  &\sum\limits_{i = 1}^{n} i^3 = \left(\dfrac{n(n + 1)}{2}\right)^2
  &&\sum\limits_{i = 1}^{n} i^4 = \dfrac{n(n + 1)(2n + 1)(3n^2 + 3n - 1)}{30} \\
  &\sum\limits_{i = 1}^{n} i^5 = \dfrac{n^2(n + 1)^2(2n^2 + 2n - 1)}{12}
  &&\sum\limits_{i = 1}^{n} i^6 = \dfrac{n(n + 1)(2n + 1)(3n^4 + 6n^3 - 3n + 1)}{42} \\
  &\sum\limits_{i = 1}^{n} i^7 = \dfrac{n^2(n + 1)^2(3n^4 + 6n^3 - n^2 - 4n + 2)}{24}
  &&\sum\limits_{i = 0}^{n} {\binom{n}{i} a^{n - i} b^i} = (a + b)^n \\
  &\sum\limits_{i = 0}^{n} i \binom{n}{i} = n 2^{n - 1}
  &&\sum\limits_{i = 0}^{n} \dfrac{\binom{n}{i}}{i + 1} = \dfrac{2^{n + 1} - 1}{n + 1} \\
  &\sum\limits_{k = 0}^{m} \binom{n + k}{n} = \binom{n + m + 1}{n + 1}
  &&\sum\limits_{i = k}^{n} \binom{i}{k} = \binom{n + 1}{k + 1} \\
\end{align*}

\subsection{Pythagorean triple}
\begin{itemize}
  \item A Pythagorean triple is a triple of positive integers $a$, $b$, and $c$ such that $a^2 + b^2 = c^2$.
  \item If $(a, b, c)$ is a Pythagorean triple, then so is $(ka, kb, kc)$ for any positive integer $k$.
  \item A primitive Pythagorean triple is one in which $a$, $b$, and $c$ are coprime.
  \item Generating Pythagorean triple
  \begin{itemize}
    \item Euclid's formula: with arbitrary $0 < n < m$, then: $$a = m^2 - n^2, \quad b = 2mn,\quad c = m^2 + n^2$$ form a Pythagorean triple.
    \item To obtain primitive Pythagorean triple, this condition must hold: $m$ and $n$ are coprime, $m$ and $n$ have opposite parity.
  \end{itemize}
\end{itemize}
